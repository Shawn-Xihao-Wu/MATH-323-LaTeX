\documentclass[12pt]{article}

% My marco file
\usepackage{mymarco}

% Begin page style
\usepackage[margin=1.0in]{geometry}
\usepackage{fancyhdr}
\pagestyle{fancy}
\setlength{\headheight}{15pt}
\rhead{36123040 Shawn Wu} % Define header
\lhead{MATH 323 HW 02} % Put assignment #
\cfoot{\thepage}

\begin{document}

% Problem 1
\begin{fproof}[Jacobson 2.3.2]
    First note that since \(R\) is commutative, then according to Jacobson on p.95, any main formulas on determinants on linear algebra over a field can be extended to that over \(R\), which means that \(\det: M_n(R) \to R\) is a homomorphism.
    Therefore, since \(AB = 1\), 
    \begin{align*}
        1 = \det(1)= \det(AB) = \det(A) \det(B) = \det(B)\det(A)
    \end{align*}
    This means that \(\det(B)\) is invertible in \(R\).
    And by Theorem 2.1 in Jacobson,
    \(B\) is invertible.
    Since \(AB = 1\), \(B\) is invertible in \(M_n(R)\), then \(B^{-1} = A\), as multiplicative inverse is unique.
    Therefore, \(BA = AB = 1\).
\end{fproof}
\newpage

% Problem 2
\begin{fproof}[Jacobson 2.4.5]

\end{fproof}
\newpage

% Problem 3
\begin{fproof}[Jacobson 2.5.2]
  We first show that \((IJ)K \subseteq I(JK)\).
  Pick any element \(x \in (IJ)K\).
  Then,
  \begin{align*}
    x = d_1c_1 + d_2c_2 + \cdots + d_{m}c_{m},
  \end{align*}
  where \(m \in \mathbb{N}\), and \(d_1, \cdots, d_{m} \in IJ\),\( c_1, \cdots, c_{m} \in K\).
  Also note that for each \(i = 1, \cdots, m\).
  \(d_i = a_1b_1 + a_2b_2 + \cdots + a_{t_i}b_{t_i}\) where \(t_i \in \mathbb{N}\), \(a_1, \cdots, a_{t_i} \in I\), \(b_1, \cdots, b_{t_i} \in J\).
  Therefore,
  \begin{align*}
    x &= d_1c_1 + d_2c_2 + \cdots + d_{m}c_{m}\\
    &= \sum_{j=1}^{t_1} a_jb_jc_1 + \sum_{j=1}^{t_2} a_jb_jc_2 + \cdots + \sum_{j=1}^{t_m} a_jb_jc_m\\
    &= \sum_{j=1}^{t_1} a_j(b_jc_1) + \sum_{j=1}^{t_2} a_j(b_jc_2) + \cdots + \sum_{j=1}^{t_m} a_j(b_jc_m).
  \end{align*}
  Note that each \(b_jc_i\) is an element of \(JK\).
  Therefore, the above sum is in the finite sum of products elements of \(I\) with elements of \(JK\).
  Therefore, \(x \in I(JK)\).
  And with a similar reasoning on the associativity of multiplication of the ring, we get that \((IJ)K \supseteq I(JK)\).
\end{fproof}
\newpage

% Problem 4
\begin{fproof}[Jacobson 2.5.3]

\end{fproof}
\newpage

% Problem 5
\begin{fproof}[Jacobson 2.6.4]

\end{fproof}
\newpage

% Problem 6
\begin{fproof}[Jacobson 2.7.2]

\end{fproof}
\newpage

% Problem 7
\begin{fproof}[Jacobson 2.7.4]

\end{fproof}
\newpage

% Problem 8
\begin{fproof}[Jacobson 2.7.9]

\end{fproof}
\newpage

% Problem 9
\begin{fproof}[Jacobson 2.7.10]

\end{fproof}
\end{document}