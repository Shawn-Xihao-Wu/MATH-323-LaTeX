\documentclass[12pt]{article}

% My marco file
\usepackage{mymarco}

% Begin page style
\usepackage[margin=1.0in]{geometry}
\usepackage{fancyhdr}
\pagestyle{fancy}
\setlength{\headheight}{15pt}
\rhead{36123040 Shawn Wu} % Define header
\lhead{MATH 323 HW 02} % Put assignment #
\cfoot{\thepage}

\begin{document}

% Problem 1
\begin{fproof}[Jacobson 2.3.2]
    First note that since \(R\) is commutative, then according to Jacobson on p.95, any main formulas on determinants on linear algebra over a field can be extended to that over \(R\), which means that \(\det: M_n(R) \to R\) is a homomorphism.
    Therefore, since \(AB = 1\), 
    \begin{align*}
        1 = \det(1)= \det(AB) = \det(A) \det(B) = \det(B)\det(A)
    \end{align*}
    This means that \(\det(A), \det(B)\) are invertible in \(R\).
    And by Theorem 2.1 in Jacobson,
    \(A^{-1}\) and \(B^{-1}\) thus exists.
    Since \(AB = 1\), \(B\) is invertible in \(M_n(R)\), then \(B^{-1} = A\), as multiplicative inverse is unique.
    Therefore, \(BA = AB = 1\).
\end{fproof}
\newpage

% Problem 2
\begin{fproof}[Jacobson 2.4.5]

\end{fproof}
\newpage

% Problem 3
\begin{fproof}[Jacobson 2.5.2]
  
\end{fproof}
\newpage

% Problem 4
\begin{fproof}[Jacobson 2.5.3]

\end{fproof}
\newpage

% Problem 5
\begin{fproof}[Jacobson 2.6.4]

\end{fproof}
\newpage

% Problem 6
\begin{fproof}[Jacobson 2.7.2]

\end{fproof}
\newpage

% Problem 7
\begin{fproof}[Jacobson 2.7.4]

\end{fproof}
\newpage

% Problem 8
\begin{fproof}[Jacobson 2.7.9]

\end{fproof}
\newpage

% Problem 9
\begin{fproof}[Jacobson 2.7.10]

\end{fproof}
\end{document}