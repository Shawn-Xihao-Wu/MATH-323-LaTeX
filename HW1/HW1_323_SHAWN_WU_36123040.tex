\documentclass[12pt]{article}

% Required packages
\usepackage[margin=1in]{geometry}
\usepackage{fancyhdr}
\usepackage{titlesec}
\usepackage{amsthm}
\usepackage{amsmath}
\usepackage{amssymb}
\usepackage{bm}
\usepackage{parskip}
\usepackage{graphicx}
\usepackage{bbm}

% \geometry{left=0.5in, right=0.5in}

% Page style
\pagestyle{fancy}

% Define header
\rhead{36123040 Shawn Wu}
\lhead{MATH 323 HW 01} % Put your actual assignment number here
\cfoot{\thepage}

% To start each section (problem) on a new page
% \titleformat{\section}[block]{\normalfont\bfseries}{\thesection}{1em}{}
% \titlespacing*{\section}{0pt}{\baselineskip}{\baselineskip}
% \newcommand{\sectionbreak}{\clearpage}


\newenvironment{fproof}[1][]
  {\begin{proof}[\ifx\relax#1\relax\else\textbf{\large #1} Proof\fi]}
  {\end{proof}}

% Redefine the QED symbol
\renewcommand{\qedsymbol}{\rule{0.7em}{0.7em}}  % Solid square

\newcommand{\set}[1]{\left\{ #1 \right\}}
\newcommand{\abs}[1]{\left\lvert #1 \right\rvert}
\newcommand{\norm}[1]{\left\lVert #1 \right\rVert}
\newcommand{\angl}[1]{\left\langle #1 \right\rangle}
\newcommand{\pare}[1]{\left( #1 \right)}
\newcommand{\brac}[1]{\left[ #1 \right]}




\begin{document}

% Problem 1
\begin{fproof}[Jacobson 2.1.1]
  Pick any functions \(f,g,h \in C\).

  \((C, +, 0)\) is an abelian group. 

  \textbf{Closure:}
  Clearly that \(f+g\) is a function as well.

  \textbf{Associativity:}
  For all \(x \in \mathbb{R}\),
  \([f(x) + g(x)] + h(x) = f(x) + [g(x) + h(x)]\) by the additive associativity of \(\mathbb{R}\). Thus, \((f + g) + h = f + (g + h)\).

  \textbf{Identity:}
  Since the integer 0 is the identity in \(\mathbb{R}\), \(f(x) + 0 = f(x) = 0 + f(x)\) for all \(x\), i.e., the zero function 0 is the identity here.

  \textbf{Inverse:}
  Note that for all \(x \in \mathbb{R}\), \(f(x) + (-1)f(x) = x - x = 0\). Thus, the additive inverse of function \(f\) is \((-1)f\) or simply \(-f\).

  \textbf{Commutative:}
  The abelianess of \(C\) follows from that of \(\mathbb{R}\). Note that for all \(x \in \mathbb{R}\), \(f(x) + g(x) = g(x) + f(x)\). Therefore, \(f + g = g + f\).

  \begin{center}
    \(\ast~\ast~\ast\)
  \end{center}

  \((C, \circ, \text{id}_{\mathbb{R}})\) is a monoid.

  \textbf{Closure:}
  Clearly that \(f \circ g\) is a function as well.

  \textbf{Associativity:}
  For all \(x \in \mathbb{R}\),
  \(((f \circ g) \circ h) (x)= (f \circ g)(h(x)) = f(g(h(x))) = f((g \circ h) (x)) = (f \circ (g \circ h)) (x)\).
  Thus, \(f \circ (g \circ h) = (f \circ g) \circ h\).

  \textbf{Identity:}
  For all \(x \in \mathbb{R}\), \((f \circ \text{id}_{\mathbb{R}}) (x) = f(\text{id}_{\mathbb{R}}(x)) = f(x) = \text{id}_{\mathbb{R}}(f(x)) = (\text{id}_{\mathbb{R}} \circ f)(x)\). Thus, \(f \circ \text{id}_{\mathbb{R}} = \text{id}_{\mathbb{R}} \circ f\).

  \begin{center}
    \(\ast~\ast~\ast\)
  \end{center}

  \((C, +, \circ)\) is not a ring as it violates the distributive law.
  Let \(f(x) = \abs{x}, g(x) = 2, h(x) = -2\) for all \(x \in \mathbb{R}\).
  Then for all \(x\),
  \begin{align*}
    (f \circ (g + h))(x) = 0 \neq 4 = (f \circ g + f \circ h)(x).
  \end{align*}




\end{fproof}

\newpage

% Problem 2
\begin{fproof}[Jacobson 2.1.4]
\end{fproof}

\newpage

% Problem 3
\begin{fproof}[Jacobson 2.2.1]
\end{fproof}

\newpage

% Problem 4
\begin{fproof}[Jacobson 2.2.4]
\end{fproof}

\newpage

% Problem 5
\begin{fproof}[Jacobson 2.2.6]
\end{fproof}

\newpage

% Problem 6
\begin{fproof}[Jacobson 2.2.7]
\end{fproof}


\end{document}
