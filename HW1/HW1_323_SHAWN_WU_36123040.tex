\documentclass[12pt]{article}
\usepackage{yassification}

% Page style
\pagestyle{fancy}
\setlength{\headheight}{15pt}
% Define header
\rhead{36123040 Shawn Wu}
\lhead{MATH 412 HW 01} % Put your actual assignment number here
\cfoot{\thepage}
\begin{document}

% Problem 1
\begin{fproof}[Jacobson 2.1.1]
  Pick any functions \(f,g,h \in C\).

  \((C, +, 0)\) is an abelian group. 

  \textbf{Closure:}
  Clearly that \(f+g\) is a function as well.

  \textbf{Associativity:}
  For all \(x \in \mathbb{R}\),
  \([f(x) + g(x)] + h(x) = f(x) + [g(x) + h(x)]\) by the additive associativity of \(\mathbb{R}\). Thus, \((f + g) + h = f + (g + h)\).

  \textbf{Identity:}
  Since the integer 0 is the identity in \(\mathbb{R}\), \(f(x) + 0 = f(x) = 0 + f(x)\) for all \(x\), i.e., the zero function 0 is the identity here.

  \textbf{Inverse:}
  Note that for all \(x \in \mathbb{R}\), \(f(x) + (-1)f(x) = x - x = 0\). Thus, the additive inverse of function \(f\) is \((-1)f\) or simply \(-f\).

  \textbf{Commutative:}
  The abelianess of \(C\) follows from that of \(\mathbb{R}\). Note that for all \(x \in \mathbb{R}\), \(f(x) + g(x) = g(x) + f(x)\). Therefore, \(f + g = g + f\).

  \begin{center}
    \(\ast~\ast~\ast\)
  \end{center}

  \((C, \circ, \text{id}_{\mathbb{R}})\) is a monoid.

  \textbf{Closure:}
  Clearly that \(f \circ g\) is a function as well.

  \textbf{Associativity:}
  For all \(x \in \mathbb{R}\),
  \(((f \circ g) \circ h) (x)= (f \circ g)(h(x)) = f(g(h(x))) = f((g \circ h) (x)) = (f \circ (g \circ h)) (x)\).
  Thus, \(f \circ (g \circ h) = (f \circ g) \circ h\).

  \textbf{Identity:}
  For all \(x \in \mathbb{R}\), \((f \circ \text{id}_{\mathbb{R}}) (x) = f(\text{id}_{\mathbb{R}}(x)) = f(x) = \text{id}_{\mathbb{R}}(f(x)) = (\text{id}_{\mathbb{R}} \circ f)(x)\). Thus, \(f \circ \text{id}_{\mathbb{R}} = \text{id}_{\mathbb{R}} \circ f\).

  \begin{center}
    \(\ast~\ast~\ast\)
  \end{center}

  \((C, +, \circ)\) is not a ring as it violates the distributive law.
  Let \(f(x) = \abs{x}, g(x) = 2, h(x) = -2\) for all \(x \in \mathbb{R}\).
  Then for all \(x\),
  \begin{align*}
    (f \circ (g + h))(x) = 0 \neq 4 = (f \circ g + f \circ h)(x).
  \end{align*}
\end{fproof}
\newpage

% Problem 2
\begin{fproof}[Jacobson 2.1.4]
  Pick any \(a + b \sqrt{-3}, c + d \sqrt{-3} \in I\).

  First, \(I\) is a subgroup of the additive group of \(\mathbb{C}\).
  Note that\((a + b \sqrt{-3}) - (c + d \sqrt{-3}) = (a-c) + (b-d) \sqrt{-3}\).
  Then,

  \textbf{Case 1:} If all \(a,b,c,d \in \mathbb{Z}\), then \(a-c, b-d \in \mathbb{Z}\).

  \textbf{Case 2:} If all \(a,b,c,d\) are halfs of odd integers, i.e., \(a = a' + 1/2, b = b' + 1/2, c = c' + 1/2, d = d' + 1/2\), for some \(a',b',c',d' \in \mathbb{Z}\).
  So, \(a-c = a'-c' \in \mathbb{Z}, b-d = b'-d' \in \mathbb{Z}\).

  \textbf{Case 3:} If only \(a,b \in \mathbb{Z}\) but \(c,d\) are halfs of odd integers, i.e., \(c = c' + 1/2, d = d' + 1/2\).
  Then \(a-c = (a-c') - 1/2, b-d = (b - d') - 1/2\) where \(a-c', b-d' \in \mathbb{Z}\). So, \(a-c, b-d\) are halfs of odd integers.

  \textbf{Case 4:} If \(a,b\) are halfs of odd integers but \(c,d \in \mathbb{Z}\). Then similar as in case 3, both \(a-c, b-d\) are halfs of odd integers.

  Therefore, in all four cases, \((a-c) + (b-d) \sqrt{-3} \in I\). Based on the subgroup criteria, \(I\) is an additive subgroup of \(\mathbb{C}\).

  Next, \(I\) is a submonoid of the multiplicative monoid of \(\mathbb{C}\).
  First note that since \(1, 0 \in \mathbb{Z}\), then \(1 = 1 + 0 \sqrt{-3} \in I\). It remains to check that \(I\) is closed under multiplication.
  Note that \((a + b \sqrt{-3})(c + d \sqrt{-3}) = (ac - 3bd) + (ad + bc) \sqrt{-3}\). Then,

  \textbf{Case 1:} If all \(a,b,c,d \in \mathbb{Z}\), then \(ac-3bd, ad+bc \in \mathbb{Z}\) as well.

  \textbf{Case 2:} If all \(a,b,c,d\) are halfs of odd integers as before. Then,
  \begin{align*}
    ac - 3bd &= (a' + 1/2)(c' + 1/2) - 3(b'+1/2)(d'+1/2) \\
    & = a'c' + a'/2 + c'/2 + 1/4 - 3(b'd' + b'/2 + d'/2 + 1/4) \\
    & = a'c'- 3b'd' + 1/2(a' + c' -3b' - 3d') - 1/2,
  \end{align*}
  which is either an integer or a half of an odd integer. Note that \(ac - 3bd \in \mathbb{Z}\) iff \(a' + c' -3b' - 3d'\) is odd, and \(ac - 3bd\) is a half of an odd integer iff \(a' + c' -3b' - 3d'\) is even. 
  
  Similarly, 
  \begin{align*}
    ad + bc &= a'd' + a'/2 + d'/2 + 1/4 + b'c' + b'/2 + c'/2 + 1/4 \\
    & = a'd' + b'c' + 1/2(a' + b' + c' + d') + 1/2
  \end{align*}
  which is either an integer or a half of an odd integer.
  Note that \(ad + bc \in \mathbb{Z}\) iff \(a' + b' + c' + d'\) is odd, and \(ad + bc\) is a half of an odd integer iff \(a' + b' + c' + d'\) is even.

  Now, if \((ad+bc) \in \mathbb{Z}\), then \(a'+b'+c'+d'\) is odd. This means that either one or three of \(a',b',c',d'\) are odd. This gives that \(a' + c' -3b'-3d'\) is also odd, as multiplying by \(-3\) does not change the parity of an integer, which in turn gives that \((ac-3bd) \in \mathbb{Z}\). And by a similar argument, \((ac-3bd) \in \mathbb{Z} \implies (ad+bc) \in \mathbb{Z}.\)


  \textbf{Case 3:} If \(a,b \in \mathbb{Z}\) but \(c,d\) are halfs of odd integers.
  Then,
  \begin{align*}
    ac - 3bd & = a(c' + 1/2) -3b(d'+1/2) \\
    & = ac' + a/2 - 3bd' - 3b/2\\
    & = (ac' - 3bd') - (a-3b)/2,
  \end{align*}
  which means that \(ac - 3bd\) is either an integer or a half of an integer.
  Note that \(ac - 3bd \in \mathbb{Z}\) iff \(a-3b\) is even, and \(ac - 3bd\) a half of an odd integer iff \(a-3b\) is odd.

  Similarly, 
  \begin{align*}
    ad+bc &= a(d'+1/2) + b(c'+1/2) \\
    & =  (ad' + bc') + (a+b)/2,
  \end{align*}
  which means that \(ad + bc\) is either an integer or a half on odd integer.
  Note that \(ad + bc \in \mathbb{Z}\) iff \(a+b\) is even, and \(ad + bc\) is half of an odd integer iff \(a+b\) is odd.

  Now, if \(ac-3bd \in \mathbb{Z}\), then \(a-3b\) is even.
  So \(a,b\) have the same parity, which means that \(a+b\) is even as well.
  This gives that \(ad+bc\) is an integer as well.
  And by a similar reasoning on parity, \(ad+bc \in \mathbb{Z} \implies ac-3bd \in \mathbb{Z}\).

  \textbf{Case 4}: If \(c,d \in \mathbb{Z}\) but \(a,b\) are halfs of odd integers. Then similar to case 3, it reaches the same conclusion.

  Therefore, in all four cases, \(ac-3bd\) and \(ad+bc\) are either both integers of both halfs of odd integers. This means that \((ac - 3bd) + (ad + bc) \sqrt{-3} \in I\), i.e., \(I\) is closed under multiplication.
  \(I\) is a subring of \(\mathbb{C}\).
\end{fproof}
\newpage

% Problem 3
\begin{fproof}[Jacobson 2.2.1]
  Let a finite domain \(R\) be given.
  Pick any nonzero element \(a \in R\). We aim to show that there exists \(a^{-1} \in R\) such that \(a^{-1}a = 1 = aa^{-1}\).
  It is suffice to show that \(a\) has both a right inverse \(a^{-1}_R\) and a left inverse \(a^{-1}_L\); if so, we have,
  \begin{align*}
    a^{-1}_R = (a^{-1}_La)a^{-1}_R = a^{-1}_L(aa^{-1}_R) = a^{-1}_L,
  \end{align*}
  i.e., the left inverse equals to the right inverse, which means the inverse \(a^{-1}\) exists.

  First note that since \(R\) is a domain, then the left cancellation law holds. To see this, assume \(ax = ay\) for some \(x,y \in R\), then \(ax - ay = 0\), which gives \(a(x-y) = 0\). Since \(a \neq 0\) and we are in a domain, \(a\) is thus not a zero-divisor, which means that \(x-y=0 \implies x = y\).

  Now since \(R\) is finite, we can enumerate all elements of \(R\) as 
  \begin{align*}
    r_1, r_2, \cdots, r_k,
  \end{align*}
  for some positive integer \(k\).
  And we claim that the following is also an enumeration of all elements of \(R\),
  \begin{align*}
    ar_1, ar_2, \cdots, ar_k,
  \end{align*}
  as it contains \(k\) distinct elements of \(R\).
  Note that they are distinct because if \(ar_i = ar_j\), then by left cancellation law established above, \(r_i = r_j\).
  Therefore, there exists \(1 \leq s \leq k\) such that \(ar_s = 1\), i.e., \(a^{-1}_R = r_s\).
  And by a similar argument as above, \(a\) must also have a left inverse \(a^{-1}_L\). 
  This proves that \(a^{-1}\) exists, which concludes the proof.
\end{fproof}
\newpage

% Problem 4
\begin{fproof}[Jacobson 2.2.4]
  We show that \(1-ba\) has both a left inverse and a right inverse, and they are equal.

  We name the inverse of \(1-ab\) as \(c\). Then, we see that
  \begin{align*}
    (1-ab)c = 1 & \implies c - abc = 1 \implies bc - babc = b \implies bca - babca = ba \\
    & \implies (1-ba)bca = ba \implies (1-ba)bca - ba = 0\\
    & \implies (1-ba)bca + (1-ba) = 1\\
    & \implies (1-ba)(bca + 1) = 1.
  \end{align*}
  Therefore, \((1-ba)^{-1}_R\) exists.
  Similarly,
  \begin{align*}
    c(1-ab) = 1 & \implies c - cab = 1 \implies ca - caba = a \implies bca - bcaba = ba\\
    & \implies bca(1-ba) = ba \implies bca(1-ba) - ba = 0\\
    & \implies bca(1-ba) + (1-ba) = 1\\
    & \implies (bca + 1)(1-ba) = 1.
  \end{align*}
  Therefore, \((1-ba)^{-1}_L\) exists.
  Since \((1-ba)^{-1}_L = bca+1 = (1-ba)^{-1}_R\), this concludes the proof.
\end{fproof}
\newpage

% Problem 5
\begin{fproof}[Jacobson 2.2.6]
  We show equivalence by proving \((1) \implies (3), (3) \implies (2)\), and \((2) \implies (1)\).

  \((1) \implies (3):\)
  Suppose \(u\) has two distinct right inverses, \(x_1, x_2\).
  Then \(ux_1 = 1 = ux_2\), which gives that \(ux_1 - ux_2 = 0\).
  Then by distributive law, \(u(x_1-x_2) = 0\).
  Since \(x_1 \neq x_2\), \(x_1 - x_2 \neq 0\), which means that \(u\) is a left zero-divisor.

  \((3) \implies (2):\)
  Suppose \(u\) is a left zero-divisor, then there exists \(x \neq 0\) such that \(ux = 0\).
  Now for the sake of contradiction, suppose that \(u\) is a unit, then \(u^{-1}\) exists. 
  Therefore,
  \begin{align*}
    x = 1 \cdot x = (u^{-1}u)x = u^{-1}(ux) = u^{-1} \cdot 0 = 0.
  \end{align*}
  This contradictions \(x \neq 0\). Hence, \(u\) is not a unit.

  \((2) \implies (1):\)
  We do proof by contraposition.
  Suppose that \(u\) has a unique right inverse, \(x\).
  Then \(ux = 1\), and \(uxu = u\), or \(uxu-u=0\).
  Therefore, \(uxu - u+ux = 1\), i.e., \(u(xu-1+x) = 1\).
  Since \(u\) has a unique right inverse, then \(xu - 1 + x = x\), which gives that \(xu = 1\).
  So, \(x\) is also a left inverse \(u\), which proves that \(u\) is a unit.

\end{fproof}
\newpage

% Problem 6
\begin{fproof}[Jacobson 2.2.7]\

\end{fproof}
\end{document}